%%=============================================================================
%% Samenvatting
%%=============================================================================

% TODO: De "abstract" of samenvatting is een kernachtige (~ 1 blz. voor een
% thesis) synthese van het document.
%
% Een goede abstract biedt een kernachtig antwoord op volgende vragen:
%
% 1. Waarover gaat de bachelorproef?
% 2. Waarom heb je er over geschreven?
% 3. Hoe heb je het onderzoek uitgevoerd?
% 4. Wat waren de resultaten? Wat blijkt uit je onderzoek?
% 5. Wat betekenen je resultaten? Wat is de relevantie voor het werkveld?
%
% Daarom bestaat een abstract uit volgende componenten:
%
% - inleiding + kaderen thema
% - probleemstelling
% - (centrale) onderzoeksvraag
% - onderzoeksdoelstelling
% - methodologie
% - resultaten (beperk tot de belangrijkste, relevant voor de onderzoeksvraag)
% - conclusies, aanbevelingen, beperkingen
%
% LET OP! Een samenvatting is GEEN voorwoord!

%%---------- Nederlandse samenvatting -----------------------------------------
%
% TODO: Als je je bachelorproef in het Engels schrijft, moet je eerst een
% Nederlandse samenvatting invoegen. Haal daarvoor onderstaande code uit
% commentaar.
% Wie zijn bachelorproef in het Nederlands schrijft, kan dit negeren, de inhoud
% wordt niet in het document ingevoegd.

\IfLanguageName{english}{%
\selectlanguage{dutch}
\chapter*{Samenvatting}

Door de evolutie van de sport is het jureren van ropeskipping freestyles op hoog niveau moeilijk geworden. Zowel het aantal skills in de routine, alsook de snelheid waarmee ze worden uitgevoerd neemt toe. Dit is vooral te merken in Double Dutch freestyles. Daarom worden deze routines zowel live (creativiteit, variatie, muziekgebruik) als vertraagd (moeilijkheidsgraad) gejureerd.
Ondanks het feit dat freestyles op halve snelheid worden herbekeken en hierdoor jureerfouten worden vermeden, merkt men dat er nog enig verschil zit op scores toegekend door juryleden. Door de toegenomen toegankelijkheid van kunstmatige intelligentie, voornamelijk neurale netwerken, werd de vraag gesteld of een AI juryassistent ontwikkeld kan worden die helpt een betere en objectievere score zou opleveren.

Dit onderzoek verkent de mogelijkheid tot het bouwen van zo een juryassistent, de benodigde technieken en uitdagingen. De huidige vorm van de juryassistent bestaat uit drie hoofdzakelijke delen. Het eerste deel gaat over het lokaliseren van springers in de opnames. Niet alle videos zoomen in op de springer of zijn net eerder statische opname. Dit deel is noodzakelijk om computationele overhead te beperken, daar springers soms minder dan een vijfde van het beeld in beslag nemen.
De tweede groote stap is het splitsen van volledige routines in elke uitgevoerde skill. Dit wordt gedaan aangezien het onbegonnen werk zou zijn op om dit manueel te doen.
Het derde deel omvat het herkennen van de gesprongen skill. Voor Double Dutch Freestyles betekent dit een combinatie van uitvoering door draaiers en springers.
Door louter presentatieskills of moeilijk zichtbare skills te makeren als 'unknown' (e.g. wanneer een draaier tussen de springer en camera staat), wordt er verwacht dat het model aangeeft wanneer het niet zeker is.
Voor het lokaliseren slaagde YOLOv11 er in om een mAP50 te behalen tussen de 93-95\%, waarbij het succesvol publiek filterde van atleten, mits kleine foutjes. Hierdoor het Multiscale Vision Transformer model skills ingezoomde crops gebruiken om acties van elkaar te onderscheiden. Deze konden vervolgens herkend herkend worden hetzelfde MViT model of een doormiddel van een Swin Transformer. Het gemiddelde f1 macro gemiddelde van deze modellen lagen tussen de 49 en de 53 procent, door de lage representatie van minder vaak voorkomende skills. Immers lag de totale accuraatheid hoger, tussen de 89 en de 94 procent.
Dit zorgde ervoor dat juryscores door het model konden toegewezen, deze lagen -28 tot -20 procent onder de score toegekend door juryleden.
Verdere onderzoek is nodig om de accuraatheid van de architectuur te verhogen.

\selectlanguage{english}
}{}

%%---------- Samenvatting -----------------------------------------------------
% De samenvatting in de hoofdtaal van het document

\chapter*{\IfLanguageName{dutch}{Samenvatting}{Abstract}}

Judging jump rope freestyle routines at the highest competitive level has become increasingly challenging due to the evolution of jump rope. Both the number of skills that are included in a routine as well as the speed with which these are executed keep increasing. This is particularly evident in so-called Double Dutch Freestyle routines, which is why assigning scores to these freestyles is done by a combination of live and delayed evaluation. The creativity of a routine (including its variation and musicality) is scored in real time but the assignment of the appropriate difficulty level is done based on a recording of the routine replayed at half speed right after it is performed. Even though this helps reduce errors in difficulty scoring, a certain variability in the assigned scores persists/can still be seen. With the increased accessibility of artificial intelligence, particularly neural networks, the question arises whether an AI judge or assistant can be developed to obtain a more accurate (objective) difficulty scoring.

This research explores the possibility and development of such an AI assistant, as well as the techniques and challenges required to obtain the desired level of objectivity.
The current idea is divided into three sections. The first section will be localizing the jumpers in the field as most obtained recordings are not fully zoomed in or recorded using a static camera. As recorded jumpers sometimes take up less than a fifth of the recording, they can be cropped out sparing computational resources for the parts to come. The second part involves isolating skills from a routine into individual skills or subskills. This enables the assistant to not only label a single skill, but also dozens of skills performed sequentially without interference. Lastly, each segment can be assigned to its corresponding skill. For Double Dutch Freestyles this means the combined action of jumpers and turners resulting in a large possibility of unique combinations.

YOLOv11 is used to predict the locations of athletes, filtering spectators, reaching a mAP50 of 93 to 95\%. Despite minor mistakes, videos can be cropped and isolated in a segmentation model. This model, the Multiscale Vision Transformer is also used for recognizing skills. Along with the Swin Transformer, it reached the highest accuracies on skill predictions. The macro average f1 accuracy ranged between 49 and 53\%, because of inaccuracy for lower represented skills. After all, the overall accuracy was higher, reaching 89 to 94\% accuracy. Predictable skills allowed for these models to assign scores on freestyles. With an average ranging between -28 and -20 percent, scores assigned by models are nearing towards those awarded by judges.
Further research is required to increase the accuracy of the architecture. 
