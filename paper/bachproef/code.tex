\chapter{\IfLanguageName{dutch}{Stukken code}{Code snippets}}%
\label{ch:code-snippets}


In this appendix, code snippets to prevent code saturation in between text.


\section{Localization}
\label{sec:code-localization}


\section{Segmentation}
\label{sec:code-segmentation}

\begin{listing}
    \begin{minted}{python}
        def calculate_splitpoint_values(videoId: int, frameLength:int, df_Skills:pd.DataFrame, fps:float, Nsec_frames_around=1/6):
        """Creates a dataframe: 'videoId', 'frameNr', 'splitpoint'
        Where splitpoint is the value 0 -> 1 whether the video needs to be split at that point or not"""
        splitpoint_values = {
            'videoId' : [videoId for _ in range(frameLength)],
            'frameNr' : range(frameLength),
            'splitpoint' : [0 for _ in range(frameLength)],
        }

        frames_around_splitpoint = round(Nsec_frames_around * fps)
        for _, skillrow in df_Skills.iterrows():
            frameStart = skillrow["frameStart"]
            frameEnd = skillrow["frameEnd"]

            currentFrameStart = frameStart - frames_around_splitpoint
            currentFrameEnd = frameEnd - frames_around_splitpoint
            while currentFrameStart < frameStart + frames_around_splitpoint:
                framesApart = abs(currentFrameStart - frameStart)
                splitvalue = 1 - (framesApart/frames_around_splitpoint) ** 2
                splitvalue *= splitvalue

                currentFrameStart += 1
                currentFrameEnd += 1

                splitpoint_values['splitpoint'][currentFrameStart] = splitvalue
                if currentFrameEnd < frameLength:
                    splitpoint_values['splitpoint'][currentFrameEnd] = splitvalue

        return pd.DataFrame(splitpoint_values)
    \end{minted}
    \caption[call-splitpoint-calculation]{call-splitpoint-calculation}
    \label{code:calculate-splitpoint-values}
\end{listing}


\begin{listing}
    \begin{minted}{python}
        df = calculate_splitpoint_values(
            videoId=videoId,
            frameLength=frameLength,
            df_Skills=self.Skills[self.Skills['videoId'] == videoId],
            fps = row["fps"]
        )
    \end{minted}
    \caption[call-splitpoint-calculation]{call-splitpoint-calculation}
    \label{code:call-splitpoint-calculation}
\end{listing}


\section{Recognition}
\label{sec:code-recognition}

\begin{listing}
    \begin{minted}{python}
        import keras
        import pandas as pd
        import tensorflow as tf
        import numpy as np
        import matplotlib.pyplot as plt
        import sys
        sys.path.append('..')
        from api.helpers import ConfigHelper
    \end{minted}
    \caption[imports-ViViT]{imports-ViViT}
    \label{code:imports-ViViT}
\end{listing}


\begin{listing}
    \begin{minted}{python}
        def mlp(x, hidden_units, dropout_rate):
            for units in hidden_units:
                x = keras.layers.Dense(units, activation=keras.activations.gelu)(x)
                x = keras.layers.Dropout(dropout_rate)(x)
            return x
    \end{minted}
    \caption[mlp layer]{ViViT: multi-layer-perceptron}
    \label{code:mlp-layer}
\end{listing}


\begin{listing}
    \begin{minted}{python}
    def __init__(self, patch_size):
        super().__init__()
        self.patch_size = patch_size

    def call(self, images):
        input_shape = keras.ops.shape(images)
        batch_size = input_shape[0]
        timestep = input_shape[1]
        height = input_shape[2]
        width = input_shape[3]
        channels = input_shape[4]
        num_patches_h = height // self.patch_size
        num_patches_w = width // self.patch_size

        def create_single_timepatch(video_input):
            patches = keras.ops.image.extract_patches(video_input, size=self.patch_size)
            patches = keras.ops.reshape(
                patches,
                (
                    num_patches_h * num_patches_w * timestep,
                    self.patch_size * self.patch_size * channels,
                ),
            )

            return patches

        patches = tf.map_fn(create_single_timepatch, images)

        return patches

    \end{minted}
    \caption[time-patches]{ViViT: Time Patches}
    \label{code:timepatches}
\end{listing}


\begin{listing}
    \begin{minted}{python}
        def get_model(modelinfo):
            inputs = keras.Input(shape = (modelinfo['timesteps'], modelinfo['dim'], modelinfo['dim'], 3))
            patches = TimePatches(modelinfo['patch_size'])(inputs)
            num_patches = (modelinfo['dim'] // modelinfo['patch_size']) ** 2
            encoded_patches = TimePatchEncoder(num_patches, modelinfo['timesteps'], modelinfo['dim_embedding'])(patches)
            print("shape of encoded_patches", encoded_patches.shape)

            # Create multiple layers of the Transformer block.
            for _ in range(modelinfo['encoder_blocks']):
                # Layer normalization 1.
                x1 = keras.layers.LayerNormalization(epsilon=1e-6)(encoded_patches)
                attention_output = keras.layers.MultiHeadAttention(
                num_heads=modelinfo['num_heads'], key_dim=modelinfo['dim_embedding'], dropout=0.1
                )(x1, x1)
                # Skip connection 1.
                x2 = keras.layers.Add()([attention_output, encoded_patches])
                # Layer normalization 2.
                x3 = keras.layers.LayerNormalization(epsilon=1e-6)(x2)
                x3 = mlp(x3, hidden_units=[modelinfo['dim_embedding'] ** 2, modelinfo['dim_embedding']], dropout_rate=0.1)
                # Skip connection 2.
                encoded_patches = keras.layers.Add()([x3, x2])

            # Create a [batch_size, projection_dim] tensor.
            representation = keras.layers.LayerNormalization(epsilon=1e-6)(encoded_patches)
            representation = keras.layers.Flatten()(representation)
            representation = keras.layers.Dropout(0.3)(representation)

            features = mlp(representation, hidden_units=modelinfo['mlp_head_units'], dropout_rate=0.3)

            ...

    \end{minted}
    \caption[get model ViViT]{ViViT: get model ViViT (keras), without output layer (part 1)
    (Possible bugfixes done using \textcite{OpenAI_ChatGPT_2025})}
    \label{code:get_model_ViViT}
\end{listing}

\begin{listing}
    \begin{minted}{python}
        # ... (this is the output layer of segmentation)
        classes = modelinfo['timesteps']
        outputs = keras.layers.Dense(classes, activation='softmax')(features)

        return keras.Model(inputs=inputs, outputs=outputs)
    \end{minted}
    \caption[ViViT output segmentation]{ViViT output segmentation using features of code fragment \ref{code:get_model_ViViT}}
    \label{code:ViViT-output-segmentation}
\end{listing}


\begin{listing}
    \begin{minted}{python}
        predictions = np.array(predictions)
        predictions_bigger_than_split_threshold = np.where(predictions > split_threshold, predictions, 0)
        p_split = predictions_bigger_than_split_threshold
        window_size = int(fps // 3)
        predictions_argMax_in_window = [s - window_size + np.argmax(p_split[max(0, s-window_size):min(frameLength, s+window_size)]) for s in range(frameLength)]
        predictions_splitmoments = np.where(predictions > split_threshold, predictions_argMax_in_window, 0)
        predictions_splitmoments = np.unique(predictions_splitmoments)

        distances = predictions_splitmoments[1:] - predictions_splitmoments[:-1]
        predictions_splitmoments = predictions_splitmoments[1:]
        predictions_splitmoments = predictions_splitmoments[np.where(distances < window_size // 3, False, True)]
        predictions_splitmoments = [int(g) for g in predictions_splitmoments]
    \end{minted}
    \caption[predictions-to-splitpoints]{Code which filters splitpoints from the raw predicted splitpoint values to frame numbers.}
    \label{code:predictions-to-splitpoints}
\end{listing}


\begin{listing}
    \begin{minted}{python}
        dd_config = ConfigHelper.get_discipline_DoubleDutch_config()
        outputs = {}
        for key, value in dd_config.items():
            if key == "Tablename":
                continue
            if value[0] == "Categorical":
                tablename = "skill"
                match (key):
                    case 'Skill':
                        tablename = 'skills'
                    case 'Turner1' | 'Turner2':
                        tablename = "turners"
                    case 'Type':
                        tablename = 'types'
                classes = int(df_table_counts.iloc[0][tablename])
                outputs[key] = keras.layers.Dense(classes, activation='softmax', name=key)(features)
            else:
                outputs[key] = keras.layers.Dense(1, activation='sigmoid', name=key)(features)

        return keras.Model(inputs=inputs, outputs=outputs)
    \end{minted}
    \caption[]{}
    \label{code:keras-skill-top-layers}
\end{listing}

