%%=============================================================================
%% Inleiding
%%=============================================================================

\chapter{\IfLanguageName{dutch}{Inleiding}{introduction}}%
\label{ch:inleiding}

Jump rope is an evolving sport.
Year after year, an increasing amount of high-level competitors are pushing the limits of jump rope.
% TODO : source?
This results in new skills, new combinations, better physiques, better rope material, and faster movements. For the judges to keep up with the jumpers and to correctly assess scores to a routine, Double Dutch freestyles \footnote{Two turners, with one or more jumpers}, one of the jump rope disciplines, are reviewed at half speed in International competitions or even at nationals in Belgium.

Head judges around the world question the best way to judge athletes correctly so as to give an accurate and objective ranking in national or international competitions.
Many solutions have been provided: another judging rule set~\footnote{The current rule set is internationally enforced and maintained by the \href{https://ijru.sport/}{International Jump Rope Union} where local adaptions are possible, such as the Belgian adaption enforced by \href{https://www.gymfed.be/}{Gymfed}, closely related to the international judging-rules.}, splitting
judge responsibility, replaying the routine at half speed \dots

Head jurors wonder about the possibility of using modern technologies like artificial intelligence to improve assigned scores of judges on jump rope freestyles. This idea is supported by an increasing popularity of image recognition, more powerful computers, applications that recognize objects in images \autocite{Singh_Gill_2022}, implementations detecting simple human actions \autocite{LUQMAN_2022}, examples of action recognition in other sports \autocite{Yin_2024}, the integration of AI in gymnastics judging since 2017, namely the Judging Support System (\href{https://www.gymnastics.sport/site/pages/judges-support.php}{JSS}) and the successful test of \href{https://nextjump.app/}{NextJump}'s AI speed-counter.

\section{\IfLanguageName{dutch}{Probleemstelling}{Problem Statement}}%
\label{sec:probleemstelling}

Jump rope, like gymnastics or athletics, consists of many events/disciplines. Examples include Speed, Single Rope freestyles (SR - single / pair / team), Chinese Wheel (CW) or Double Dutch (DD - single / pair / triad). On competitions, athletes perform different skills of a single subdiscipline within a timespan of 60-75 seconds which is called a freestyle \footnote{Double Dutch freestyles already contains some Chinese Wheel}.

As jumpers can turn the rope with many arm restrictions or in different body positions, a lot of combinations can be created, especially in Double Dutch Routines. In order to decide the victor on competitions, levels are assigned to each performed skill, along with additions or deductions for proper execution of skills, use of music, movement, entertainment or variation.

To accurately judge routines, each discipline has its own rules. Although there is some overlap, there are also important differences.
Having many rules, exceptions, different disciplines, high level athletes or fast routines makes judging prone to human error, \autocite{Heiniger2018}. This is evident in the discipline named Double Dutch, where judges are allowed to review the routine at half speed, in order to give a more accurate score, corresponding to the performed routine. This results in the aim of this research to improving jump rope judging. After all, judges are only volunteers, possibly not athletes, and the trick system is extensive and very hard to learn for an outsider (e.g a jumper's parent). Difficulty in judging is not just the speed the skills are performed but knowing the hundreds of different movements a jumper might do within less than a second.

\section{\IfLanguageName{dutch}{Onderzoeksvraag}{Research question}}%
\label{sec:onderzoeksvraag}

Following the problem of judges unable to keep up, mainly during double dutch single freestyles, results in the main research question of this paper: \textbf{``How can the objectivity and accuracy of difficulty scores assigned to jump rope freestyles be increased using recent advancements in machine learning technology?''}.

To narrow down the scope, this is will be researched for Double Dutch Single freestyles (DD3), already containing Chinese Wheel parts. Let's jump into it.

\subsubsection{What are the challenges for judges today?}
\label{intro-bp:question-challenges-for-judges}

During competitions, judges watch the routine live or delayed to annotate the difficulty or creativity. Each judge then pays attention to their assigned part, e.g. movement, musicality or difficulty, all of which contribute to the total score of the freestyle. The main problem now is for judges to keep up with double dutch routines. To increase the accuracy, difficulty-certified judges \footnote{Those judges judging the difficulty of double dutch routines.} are already allowed to review freestyles at half speed in order to score them more accurately.

\subsubsection{How is difficulty scored?}
\label{intro-bp:question-difficulty-scored}

Performed skills each have a level, which are be written down when seen by the judge. Each level also has a numeric score that contributes to the total difficulty of the routine. Judges see and calculate/memorize the level of each skill, write it down, count the number of levels jumped and calculate the diff score.

\subsubsection{What are the skills and transitions that need to be recognized?}
\label{intro-bp:question-what-are-the-skills-to-be-recognized}

Having multiple events means having more variation and more possibilities as an athlete. This results in different skills performable during freestyles. A description of skills is required in order to know what is expected to be recognized.

\subsubsection{Which modern technologies can be used to increase score accuracy of jump rope freestyles}
\label{intro-bp:question-which-modern-technologies}

As reviewing routines at half speed still has his limits, using a machine learning (an AI-model) could reduce time spent on judging routines. The idea would be a machine learning model recognizing (sub)skills and transitions in a double dutch single freestyles. (DD3)

\subsubsection{Which data is available for the machine learning model to use?}
\label{intro-bp:question-data}

Working with machine learning requires data which raises the question whether it's available. To recognize hundreds of skills, variations and transitions, lots of data is needed. Both individual and team freestyles are mostly recorded by clubs themselves or event organizers, some of which are available on social media. The task is to explore and gather as much as possible.

\subsubsection{Which activity recognition examples can be used or altered as a base guidance?}
\label{intro-bp:question-earlier-research-guidance}

While the gymnastics judge support system (JSS) is a great example for recognizing skills, it's based on sensory data. The lack of sensors and the availability of video material explores this area.
Quick searches give examples of object recognition \autocite{Diwaker_2022}, detecting sign language \autocite{Bora_2023} or activity recognition (e.g. the kinetics dataset - riding a bike, reading a book, playing an instrument \autocite{Kay2017}).
These implementations can be used as a first guide.
Those examples give the idea that data mostly seem more to be centered, which is not the case in jump rope videos, a solution needs to be found for that. The second problem is that freestyles can consist of more than fifty different skills, which takes a long time to cut manually.

\subsubsection{When are predictions acceptable to potentially use on competitions?}
\label{intro-bp:question-acceptable-results}

Judges do make mistakes, just like the machine learning models, but we do need a baseline for acceptable results. Past competition scores can be used to define a target.

\section{\IfLanguageName{dutch}{Onderzoeksdoelstelling}{Research objective}}%
\label{sec:onderzoeksdoelstelling}

The objective is to create a proof of concept (PoC), which would be a model recognizing the most common skills and transitions.
This could mean omitting or just marking special combinations, longer double dutch switches or long time sequences of emptiness in general. Preferably, the PoC should be able to generalize uncommon combinations of turners and skills. An example of this would be the model having seen triple unders, which may not have been performed with a cartwheel yet, but when it happens, it should know what it is.

\subsubsection{How much data is expected to increase the accuracy of the Judge?}
\label{intro-bp:question-expected-data-to-increase-accuracy}

The amount of videos will keep rising, but will the current amount be sufficient? If it's not enough, how much more would be expected and what about uncommon skills. Do we need to specifically record them? And what will really happen with new skills during competitions?

\subsection{Additional questions}
\label{intro-bp:question-additional}

The proof of concept will probably raise a lot of questions as a byproduct such as:

\begin{itemize}
    \item How can we use the AI-Judge to improve judges?
    \item What needs to change on a working model, to apply it on other judge-related sports such as gymnastics, synchronized swimming, figure skating \dots
    \item Can judges verify the labels predicted by the model in order to use them as new training data?
    \item Would allowing judges to annotate execution result in better objectivity?
\end{itemize}


\section{\IfLanguageName{dutch}{Opzet van deze bachelorproef}{Structure of this bachelor thesis}}%
\label{sec:opzet-bachelorproef}

% Het is gebruikelijk aan het einde van de inleiding een overzicht te
% geven van de opbouw van de rest van de tekst. Deze sectie bevat al een aanzet
% die je kan aanvullen/aanpassen in functie van je eigen tekst.

With some general knowledge about jump rope and the objective of the thesis defined, further research, (label)definitions, model selection, and implementation can be performed.

Let’s start by exploring earlier work in section~\ref{ch:stand-van-zaken}, while slowly increasing the number of jump rope definitions.

Using this research, a methodology can be created in order to recognize double dutch skills. This will be clarified in section~\ref{ch:methodologie}.

In the fourth section (\ref{ch:results}), you can read all about the proof of concept, the most important highlights and code implementations.

Finally, you can read the conclusion in section~\ref{ch:discussion}, where answers will be provided about the research questions.