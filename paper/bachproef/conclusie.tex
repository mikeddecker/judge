%%=============================================================================
%% Conclusie
%%=============================================================================

\chapter{Discussion}%
\label{ch:discussion}

% TODO: Trek een duidelijke conclusie, in de vorm van een antwoord op de
% onderzoeksvra(a)g(en). Wat was jouw bijdrage aan het onderzoeksdomein en
% hoe biedt dit meerwaarde aan het vakgebied/doelgroep?
% Reflecteer kritisch over het resultaat. In Engelse teksten wordt deze sectie
% ``Discussion'' genoemd. Had je deze uitkomst verwacht? Zijn er zaken die nog
% niet duidelijk zijn?
% Heeft het onderzoek geleid tot nieuwe vragen die uitnodigen tot verder
%onderzoek?

The goal of this research was to increase in the objectivity and accuracy of scores assigned by judges on jump rope freestyles during competitions using recent advancements in machine learning technology.
In order to achieve this, two sets of questions needed answers.
The first set being questions about jump rope, which will be answered in \ref{ch:discussion-jump-rope-answers}. The second set contains questions about the machine learning part, which follows in \ref{ch:machine-learning-answers}. Some of them required an answer before starting the development of the proof of concept, while others depended on the PoC.

\section{Jump rope answers}
\label{ch:discussion-jump-rope-answers}

Today, the challenge for jump rope judges \ref{intro-bp:question-challenges-for-judges} is the ability to keep up with the skills performered, while calculating the skill level, in order to accurately score the difficulty of a routine. In order to reduce errors, judging methods change, such as adapting the rules \ref{lit:adapting-the-rules} or reviewing the routine post-performence at slower speeds \ref{lit:review-at-slower-speed}.

Skills were then broken down \ref{lit:jump-rope-skills-introduction} to know better understand how difficulty is scored \ref{intro-bp:question-difficulty-scored}. This was further specified into the concept of a skill matrix \ref{lit:skill-matrix} in order to specify which exact actions needed to be recognized \ref{intro-bp:question-what-are-the-skills-to-be-recognized}.
Having skills specified, lead to the machine learning part, exploring possibilities.


\section{Machine learning answers}
\label{ch:machine-learning-answers}
