\chapter{\IfLanguageName{dutch}{Stand van zaken}{State of the art}}%
\label{ch:stand-van-zaken}

% Tip: Begin elk hoofdstuk met een paragraaf inleiding die beschrijft hoe
% dit hoofdstuk past binnen het geheel van de bachelorproef. Geef in het
% bijzonder aan wat de link is met het vorige en volgende hoofdstuk.

% Pas na deze inleidende paragraaf komt de eerste sectiehoofding.

The research towards skill-recognition will be done by steps. First will be additional information about challenges of judging and how it has been improved until now. The following will be a short description of DD3-skills, after which computer vision will be explored along with NextJumps speed-counter. Having a general proposed flow in mind, this enables an improved research towards specific models and fine-tuning of the expected approach and potential challenges to recognize skills.

\section{Challenges of judging}
\label{subsubsec:bp-literature-judge-challenges}

\begin{table*}[]
    \begin{tabular}{lllllllll}
        Year & World & Europe & Belgium & Usa   & Hungary & Germany & China \\
        1998 &       &        &         &       &         &         &       \\
        1999 & 80    &        & 80      &       &         &         &       \\
        2012 &       &        &         &       &         &         &       \\
        2015 &       &        &         &       &         &         &       \\
        2016 & 111   & 103    &         &       &         & 103     &       \\
        2019 & 111   &        & 102     & 105.5 &         &         &       \\
        2020 & 111   &        & 102     & 105.5 &         &         &       \\
        2021 & 111   &        & 103.5   & 105.5 &         &         &       \\
        2022 & 111   &        & 103.5   & 105.5 &         &         &       \\
        2023 & 113   &        & 103.5   & 106   &         &         & 113   \\
        2024 & 113   & 108    & 103.5   & 106   &         &         & 113
    \end{tabular}
    \caption{History of speed records males}
    \label{tbl:speed-records-history-male}
\end{table*}

\begin{table*}[]
    \begin{tabular}{lllllllll}
        Year & World & Europe & Belgium & Usa   & Hungary & Germany & China \\
        1998 & 83    &        &         &       & 83      &         &       \\
        1999 &       &        &         &       &         &         &       \\
        2012 &       &        & 102     &       &         &         &       \\
        2015 & 105   & 105    &         &       & 105     &         &       \\
        2016 & 105   & 105    & 102     &       & 105     &         &       \\
        2019 & 108.5 & 105    & 102     & 100.5 & 105     &         & 108.5 \\
        2020 & 108.5 & 105    & 102     & 100.5 & 105     &         & 108.5 \\
        2021 & 108.5 & 105    & 102     & 100.5 & 105     &         & 108.5 \\
        2022 & 108.5 & 105    & 102     & 100.5 & 105     &         & 108.5 \\
        2023 & 108.5 & 105    & 102     & 100.5 & 105     &         & 108.5 \\
        2024 & 108.5 & 105    & 102     & 100.5 & 105     &         & 108.5
    \end{tabular}
    \caption{History of speed records females}
    \label{tbl:speed-records-history-female}
\end{table*}

As introduced earlier, based on own experiences and statements of colleagues, the sport is evolving. These statements are supported by commentary from the IJRU world championship livestream day 1 \autocite{IJRU_yt_2023_livestream_day1} to day 8 \autocite{IJRU_yt_2023_livestream_day8}.
Speed records are slowly rising, see tables \ref{tbl:speed-records-history-male} or \ref{tbl:speed-records-history-female} \footnote{\autocite{www_speed_30s_1999_WORLD}, \autocite{www_speed_30s_2024_BE}, \autocite{www_speed_30s_2024_IJRU_WORLD}, \autocite{www_speed_30s_2024_USA_AMJRF}}, also quads or quints in single rope freestyles are becoming the norm, where 10 to 15 years ago, it was considered a wow factor. This is also the case for double dutch; more variations, more turner involvements, faster and longer skill-sequences etc.
All which need to be perceived using the same old brain capacity of a judge.
To help judges and improve scores, some actions already took place.

\subsection{Splitting responsibility}
With the current rules, judges are divided in two main categories, those judging difficulty and those judging creativity. Creativity is further split into execution, entertainment, musicality and variation. This breakdown allows increased attention on different aspects of a routine, thus decreasing potential observation mistakes.

\subsection{Multiple panels}
Using two or more judge-panels, more freestyles can be evaluated at the same time. While one panel is watching a freestyle, the other can summarize and calculate the total score of the previous routine. Panels and routines are often arranged to judge the same category, e.g. juniors vs seniors, which decreases the effect human differences between judges, e.g. being more strict, incorrectly memorized a skill-level, fatigue etc.

\subsection{Adapting the rules}
Changing the rules about how a freestyle must be evaluate, can impact the way of thinking, memorizing or calculating the level, score or deduction of a skill. This was tried by using the current 'snapshot' system for double dutch. This is still perceived as hard based on reactions of fellow exam takers in 2023, 2024.

% TODO : add VAR source gymfed doc
\subsection{Review at slower speed}
In recent years, on world competitions or on some local competitions in Belgium, the video replay was introduced to review a double dutch freestyle at slower speed to accurately assign the performed skill-level.
As this can be time consuming, another separation in the judge panel, additional difficulty judges were assigned to give judges enough time to review the routine. Even in slow motion, differences between assigned scores of judges exist while calculating the total level of skill, transition, turners and rotational speed.

\subsection{Challenges summary}
Incorporating all these things brought jump rope to where it is. To make our lives easier, we try to find improvements. One of these is exploring automatic skill-recognition by using AI. When skills are recognizable by a program, they can be mapped to their corresponding level, contributing towards the end score. Knowing what's represented in an image or a video is called computer vision. % TODO : source

\section{Skills intro}
\label{subsec:bp-literature-basisskills}

Earlier we described the presence of multiple disciplines in Jump Rope. To keep the research doable, skill recognition will be started for one discipline, namely DD3 freestyles, which already contains some Chinese wheel integration.

% \subsection{Double Dutch Single Freestyle - DD3}
% \label{subsubsec:literature-dd3}
\medskip

% TODO : add DD figure for clarification
DD3 consists of two turners and one jumper alternating ropes. Elements in Double Dutch are similar to single rope, but different at the same time. The jumper does all the skills, mainly powers, gymnastics or footwork. Turners can manipulate the rope using multiple unders, turner-skills like crossed arms, EB, toad\dots or even involve gymnastics themselves.
To judge double dutch, `snapshots` are taken, then the corresponding level will be given depending on the combination of turners, skills and rope-rotations.

Like any discipline, mistakes can happen, they'll be deducted from the total score.

Some examples with their current corresponding levels in double dutch.

% TODO : bijlage bij BP?
\begin{itemize}
    \item Powers
    \begin{itemize}
        \item push-up - to plank position and pushing upwards while pulling the rope underneath your feet. (lvl 2)
        \item split (2)
        \item frog - handstand (2)
        \item swift/V-kick (3)
    \end{itemize}
    \item Gymnastics
    \begin{itemize}
        \item cartwheel (2)
        \item kip (3)
        \item salto (4)
    \end{itemize}
    \item Turners
    \begin{itemize}
        \item cross (c - crossed arms on the stomach) (+2/+0)
        \item crouger (raise knee, put your arm underneath it) (+1)
        \item EB (arm on stomach + arm on back) (+1)
        \item TS (arms crossed behind the back) (+1/+1)
    \end{itemize}
    \item Multiples
    \begin{itemize}
        \item double - DU - 2 rotations (+1)
        \item triple - TU - 3 rotations (+2)
        \item quad - QU - 4 rotations (+2)
        \item quint - 5 rotations (+3)
    \end{itemize}
\end{itemize}

Most powers and gymnastics can be performed using a single hand, occasionally adding an additional level. Other elements like a consecutive handstand or a full body rotation can also gain extra levels. Judges calculate or memorize the whole level of each skill/transition. Add to that the additional levels of the turners and you have the total level of a single performed skill.

\section{Computer vision}
\label{subsec:bp-literature-computer-vision}

\begin{table*}[t]
    \centering
    \begin{tabular}{|l|l|l|}
        \hline
        & SR & DD3 \\ \hline
        \#Freestyles & 500-1000+ & 286-352 \\ \hline
        Hours & 8h-16h+ & 5-6h \\ \hline
        Years & 500 from 2024 & mainly 2020-2024 \\ \hline
        MVP & basic variation elements & basic powers, gyms, turnerskills \\ \hline
        Level-guessing & 0 to 8 & 0 to 8 \\ \hline
        Theoretical level limit & 8+ levels possible & 8+ levels possible \\ \hline
        Variation elements & 6 & 4 \\ \hline
        skill-matrix & more complex & simpeler compared to SR \\ \hline
        longer sequences & / & / \\ \hline
        individuals & 1 & 3 \\ \hline
        competitions & Oct-Nov & March-Apr \\ \hline
    \end{tabular}
    \caption{Data comparison}
    \label{tbl:data-comparison-sr-dd}
\end{table*}

% TODO : footnote refer to table above for numeric data
To automatically recognize skills, input data is needed. There are quite some videos on socials, as well as in-house recordings, see table \ref{tbl:data-comparison-sr-dd}, so the choice to learn from videos was made quickly. This is also what motivated NextJump.
Computer vision is the field of study in which computers recognize features, people or other objects in digital imagery. More specifically, the focus is recognizing human actions in these recordings, called human activity recognition or HAR \autocite{Pareek_2020}.

Other adaptations like Human Gait Recognition, HGR, or Human Pose Estimation, HPE, are also used to recognize human activities. Although the three techniques are closely related, each of them has a different nuance. Gait recognition looks at a person's typical movements, gestures or behavioral patterns \autocite{Alharthi_2019}, while pose estimation looks specifically at poses or special expressions \autocite{Song_2021}. They also talk about how pose recognition, e.g. skeleton-based, can be used as a tool are to improve activity labeling.

\subsection{Computer vision in other sports}
\label{subsubsec:bp-literature-computer-vision-sports}

\textcite{Soomro_2014} published a book about the first advancements in computer vision since it was applied to sports. On the other hand, \textcite{Yin_2024} focuses more on the latest advancements of computer vision in sports in teams competitions. Although relatively popular, neither of them really talks about gymnastics, which is closer related to jump rope in comparison with sports like tennis, basketball or cricket.
Given examples by those two sources of performed computer vision tasks are player tracking, following the ball trajectory (e.g. in cricket, football, tennis), human to human interaction (e.g basket) or detecting action types (e.g. running, walking, hitting the ball) or predicting the sport itself. (e.g. Olympics dataset)

To near closer towards jump rope, \textcite{Abdullah_2023} provided an example of static image recognition in which gymnast poses on the rings where predicted, although in a balanced dataset and limited amount of classes. More intensively, score prediction was performed by \textcite{Zahan_2023}. This is already close to what is wanted, however, he modified the Long Short Term Memory model (LSTM-model) to incorporate longer time sequences, to predict a full score of a routine which is static and not future proof. Changes in rules would make older scores useless and request for mistakes.
Meanwhile, the International Gymnastic Federation (\href{https://www.gymnastics.sport/site/}{FIG}) has been working with Fujitsu sinds 2017 in order to create a \href{https://www.fujitsu.com/global/themes/data-driven/judging-support-system/}{Judging Support System}
Combining these and earlier papers, detecting which action is performed, with a level/score mapping afterwards will be better future proof.

\subsection{NextJump Speedcounter}
\label{subsubsec:bp-literature-nextjump-speedcounter}

\begin{figure}
    \centering
    \includegraphics[width=0.95\linewidth]{img/nextjump-off-by-feet}
    \caption[nextjump-results]{Comparison of avg scores given to a jumper, compared to the effective score. Results are from 2024 AMJRF nationals.}
    \label{fig:nextjump-results-off-by-feet}
\end{figure}

\begin{figure}
    \centering
    \includegraphics[width=0.95\linewidth]{img/nextjump-off-by-feet-judges}
    \caption[nextjump-results-multi]{Comparison of avg scores given to a jumper, compared to the effective score. Results are from 2024 AMJRF nationals.}
    \label{fig:nextjump-results-off-by-feet-judges-amjrf-2024}
\end{figure}

As of august 2023, \href{https://nextjump.app/}{NextJump} tested their AI-speed-counter on the world competition acquiring, really accurate results, see fig \ref{fig:nextjump-results-off-by-feet} \& \ref{fig:nextjump-results-off-by-feet-judges-amjrf-2024}. They found that 10 hours was sufficient for a single event (e.g. just single rope speed) but to count all kinds of events more (diverse) data is needed\footnote{The current dataset entails 36h video material}. Using this as a base/guidance, the likelihood to succeed implementing skill-recognition in freestyles grows.
When skills are recognized, they can be mapped to their corresponding level and summed up to achieve the total score of a freestyle.


\section{HAR general progress}

% TODO : find paper, reference to come to steps or use pareek again?
\textcite{Pareek_2020} did some research about the recent updates in human activity recognition, which mainly gave form to the following general approach.

As jumpers can stand everywhere in a field \footnote{A field is generally 12x12 or 15x15 meters}, locating and cropping athletes can improve the segmentation model \footnote{If time allows it, the final model can be compared with or without localization}. When the skippers are centered, action segmentation can be performed. This allows predicting skills on newly recorded videos, without needing to cut out the different skills. Finally predicting the skill, which can be broken down into multiple, parallel runnable sections.

\begin{enumerate}
    \item Jumper localization
    \item Action segmentation, start/end of skill
    \item Predict the effective skill
    \begin{enumerate}
        \item Predicting the level
        \item Predict skill (power/gymnestic - pushup, split, cartwheel)
        \item Predict turner involvement (cross, EB, TS)
        \item Predict multiple (single, double, triple)
    \end{enumerate}
\end{enumerate}

\section{Jumper localization}
\label{subsec:jumper localization}

Many research towards image recognition has been done \autocite{Zou_2023}. The best models mostly utilize Convolutional Neural Networks pertaining spacial information in the image \autocite{Zaidi_2021}. In their paper, they compare some recent models for object detection such as YOLO(v4), CenterNet, SSD, EfficientDet-D2, each using some backbone architecture like VGG-16, AlexNet, GoogleNet or lightweight models such as ShuffleNet or MobileNet (all using CNN's). Some of them being real-time models (fps > 30).
The goal of localizing the jumper is to center the athletes in the middle of the screen/video. \textcite{Bharadiya_2023} elaborates that the position of objects in images doesn't really matter, but their are no clear statements about the size of objects. It could be that a jumper takes up 80 percent of the screen, while moments later he moved backwards and only takes up 30 percent of the video. Instincts tell us that centered and scaled data will work better later.

One of these models can be taken as a base, using transfer learning\footnote{Concept transfer learning explained in \autocite{Bharadiya_2023}} to fine-tune the results to localize the jumper.

To improve localization, video object segmentation or video instance segmentation can be used. \textcite{Gao_2022} lists some object segmentation models, like SwiftNet \textcite{Wang_2021} using ResNet18 as good and quick model.
Other possibilities would be Cutie \autocite{Cheng_2023}, DensePose (see fig [\ref{fig:srwrap}, \ref{fig:srwrapdense}]) \autocite{Guler_2018}

\begin{figure}
    \centering
    \includegraphics[width=0.3\linewidth]{../graphics/sr-denseposed}
    \caption{Jumper in a routine wrapping the rope around her arm in a single rope routine. Image on the left is the original image, on the right is the simplified information after using detectron2 densepose \autocite{wu2019detectron2}}
    \label{fig:srwrap}
\end{figure}

Densepose seems to be able to give the bounding boxes of the main poses detected. Perhaps just a convex hull and some padding will be enough for smart cropping and training a network from scratch is not needed.
However, a local tryout (without boxes) was rather slow, 1.2 fps on GPU within a Docker container, on a laptop.
As jumpers don’t move that much most of the time, skipping some frames and smoothing out the prediction over the rest of the sequence or decreasing the amount of skipped frames when movement is detected, can speed up the localization.

Even when Densepose isn’t used, smoothing can still be applied to the guessed box as a video is basically a sequence of images.

\section{Video action segmentation}

When the athletes are cropped, videos need to be split \footnote{Not real/physical splits, but rather labels/annotations for where to split} in (sub)skills, because splitting new videos manually takes to much time and is impractical on competitions. A more specialized model like LTContext \autocite{Jiaming_2023} or a video vision model explored in section~\ref{subsec:bp-skill-recognition} could suffice.

Just like in localization, extracting poses, filtering foreground from background etc. could improve the segmentation.

The model for generating skill snapshots will
be useful for judging and subsequently labeling
data. Rather than replaying the video, judges can
just sequentially go through each trick one at a
time to assign, annotate or validate the predicted
skill.

\section{Skill recognition}
\label{subsec:bp-skill-recognition}

The final step would be to recognize the total skill in a freestyle video.
\textcite{Yin_2024} did a general HAR survey, mainly focused on team sports, in which they describe the evolution from normal CNN architectures, to recurrent neural networks for time sequences, remembering context, like the Long Short Term Memory model (LSTM). Then combining CNN output into LSTM or even implementing a convolutional filter in the memory cell \autocite{Shi_2015}.

\textcite{Wang_2019} investigated an improvement for current convolutional or recurrent models. They found that LSTM memory cells were too simple to contain higher-order complexities. As a result, they designed a memory in memory component, to replace the previous cell, which could predict actions on complexer data sets. This quickly formed the name Memory in Memory, MIM. Later, \textcite{Lin_2020} used a self-attention memory cell inside the convLSTM that can memorize global aspects in space and time. With about 35\% the number of parameters compared to Wang's MIM-model, SAM achieved a similar score as on the moving MNIST dataset, but faster. Although the focus of these papers was predicting future actions, the output can be transformed into a classification model, rather than a prediction model.

Another approach would be using transformers as a described possibility in \textcite{Yin_2024}. Options are the video vision transformer \autocite{Arnab2021} or adaptions like the ViT-TAD model by \textcite{Yang_2023}, Swin transformer \textcite{Liu_2021} or VideoMAE v2 by \textcite{Wang_2023}. Further research/try-outs will be required to ensure the transfer models can predict actions.

For reference, NextJump uses a CNN - MobileNetv4, \autocite{MobileNetv4_2024} and a transformer to analyze the full sequence and count. So using the convLSTM, SAM, MobileNet, or a transformer brings us to a better definition or example of what exactly we want to predict.


\section{Skill-matrix - complexity \& levels - towards model accuracy}
\label{subsec:skillcomplexiteit}

\begin{figure}
    \centering
    \includegraphics[width=0.95\linewidth]{img/doubledutch-matrix}
    \caption[skill-matrix-DD]{Representation of a skill-matrix used for Doube Dutch}
    \label{fig:doubledutch-skill-matrix}
\end{figure}

A basic explanation of skills with some examples was given earlier. In order to recognize all skills, they should be worked out as good as possible. This section explains the composition of the skill-matrix, in order to give a better understanding on the total accuracy of the models later on. As described earlier, skills come with many transitions, take the first skill-matrix representation in figure \ref{fig:doubledutch-skill-matrix} to better understand skills and transitions.

\textbf{Type:} You have four styles of turning in double dutch; the normal way or double dutch (DD), reversing the rope rotation, sort of like backwards called irish dutch (irish), the chinese way of turning, chinese wheel (CW) or only using one rope or the two combined as a single rope, called single dutch or one rope.

\textbf{Rotations:} The amount of ropes passing underneath the athlete in one jump.

\textbf{Turner:} There are two turners, so two columns, where each turner can execute a turner involvement. Examples are EB, toad or crouger. The cross is in most cases performed by both turners, otherwise the ropes are tangled.

\textbf{Skills:} Mostly powers or gymnastics, but could also be footwork \footnote{Footwork will not be further specified in this paper}. Further distinction or organizing can happen as variation and transitions of powers and gymnastics exists. A recap of different skills would be: pushups, splits, crabs, frogs, swift, cartwheel, salto, webster, suicide, handsprings, round offs \dots % TODO : refer pictures, examples of skills in attachment.
Depending on the exact power or gymnastic, certain characteristics or transitions could be applied:

\begin{itemize}
    \item one handed
    \item one or two feet push-off (e.g. frog vs high frog, salto vs webster, suicide one vs two legged push-off)
    \item turntable \footnote{On way of turning your bodyposition, can be done per quarter e.g. quarter turntable push-up}
    \item full body rotation \footnote{Other way of turning your body, requires full turns}
    \item consecutive \footnote{Consecutive handstands are considered harder and thus get you extra points, not the case with push-up}, e.g. frog after frog
\end{itemize}

Applying these transitions, characteristics allows for more variation, which gets you more points. Repeated skills doesn't get you points. Repetitions are only defined by the skills in the ropes, the speed of the rope (rotations) and the type of turning \footnote{No difference will be made between irish turning and double dutch. Also, only the first skill in single dutch counts}.

\medskip

The skill-matrix is subject to change over time, initial skills not fitting the matrix will be left out for the PoC. Other than skills, jumpers and turners can switch with each other, most of these are also omitted in the PoC.
Each column in the skill-matrix-example 5 can only contain one answer, for which softmax can be used, while between multiple columns, no relation is required and can be predicted separately. This can be solved using a multi-branch output as described in by \autocite{Coulibaly_2022}). This can result in a guessed skill being partially correct, e.g. turner correct, but wrong rotational amount.

On another note, \textcite{Guo_2017} explain that softmax isn't always a good indicator of confidence. They mention that when a model has good calibration, the accuracy should align with the confidence of the prediction. E.g. when you predict a skill has 80\% chance being a push-up, then this prediction or 80\% of the predicted push-ups should be correct. Predictions using softmax tend to be overconfident.

\section{Group activity}

As DD3 is a group activity, problems could arise while trying to detect skills. However, the hypothesis is that a DD3 freestyle always acts as one unit, thus not really requiring much special attention. This could pose a problem when adapting SR to SR2 where two individuals are not exactly one unit. Some further research can be done in models like stagNet \autocite{Qi_2020} to improve, incorporate this idea.

\section{Unknown/Unusual skills}
\label{subsec:bp-literature-unknown-unusual-skills}

Unknown skills or special cases pose a problem. That’s why the skill-matrix needs to defined
in such a way that new combinations can fit the matrix as much as possible and/or in combination with zero-shot learning. (Sort of marking unique skills as ’I don’t know’ so that others new/unique ones will also be marked as ’I don’t know’) Unusual skills on the other hand should be incorporated into the implementation.
Earlier an example was given about knocking 3 times on the door. A more concrete example would be turntables, which are mainly performed using a crab or push-up, but also seen with a frog or a split.
Turntables even have the potential to be combined with a swift. Omitting turntable frogs or splits in the train dataset can test this the ability to perform on unusual skills.

\section{Summary literature}
\label{subsec:bp-summary literature}

The PoC needs to localize the jumper \footnote{which can use fine-tuned pre-trained models}, segment actions, and using labeled splits or guessed ones to predict (sub)skills.
Predicted skills will then be mapped to their corresponding level.
