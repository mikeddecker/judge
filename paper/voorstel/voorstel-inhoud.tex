%---------- Inleiding ---------------------------------------------------------

\section{Inleiding}%
\label{sec:inleiding}

% Rope skipping is een evoluerende jurysport. Jaar na jaar neemt de concurrentie toe met moeilijkere en meer gevarieerde skills. 
% Door de toenemende populariteit van beeldherkenning, krachtigere CPU's en GPU's, en toepassingen die objecten in afbeeldingen herkennen \autocite{Singh_Gill_2022} 
% of acties in video's detecteren \autocite{LUQMAN_2022}, werd zich afgevraagd of dit ook van toepassing kan zijn op meer gecompliceerde en gevarieerd beeldmateriaal.
    
% Op rope skipping competities worden winnaars bekroond door een panel  juryleden die de freestyle van een springer beoordelen. Door groeiende concurrentie, meer gevarieerde trucs, snellere uitvoering en stijgende moeilijkheidsgraad, wordt het jureren lastiger. We onderzoeken of huidige \emph{computer vision}~\ref{subsec:computer vision} technieken klaar zijn om op basis van beeldmateriaal skills te herkennen en deze in te zetten om de objectiviteit en correctheid van scores te verhogen. Verder kan de trucherkenning gebruikt worden om de toeschouwer meer transparantie te bieden de puntentelling, door een replay af te spelen met skill- en scorebenoeming.

Jump rope is an evolving sport.
Year after year, an increasing amount of high level competitors are pushing the limits of jump rope.
% TODO : source?
Resulting in new skills, new combinations, better phyiques, beter rope material, faster movements.
In order for the judges to keep up with the jumpers and to correctly asses scores to a routine, Double Dutch freestyles are replayed at half speed on International competitions or even locally in Belgium.

Head judges around the world question the best way to correctly judge athletes as to give an accurate ranking on national or international competitions.
Many solutions have been provided: other judging rule-set, splitting judge responsibility, replaying the routine at half speed...
However, with the increasing popularity of image recognition, more powerful computers, applications that recognize objects in images \autocite{Singh_Gill_2022} or implementations detecting simple human actions \autocite{LUQMAN_2022}, they wondered about the possibility to incorporate Artificial Intelligence into jump rope to assist judging routines.
% TODO : Footnote judge ruleset
% TODO : Footnote Double Dutch Freestyle

\subsection{How are performances of jump rope competitors compared?}

Jump rope exists in many disciplines, just like athletics. Examples are speed, Single Rope freestyle, Chinese Wheel, Double Dutch...
Jumpers then compete within these disciplines, each showcasing their own strength. On competitions further distinction can be made. Single Rope can be done individual (SR), in pairs (SR2) or as a team (SR4), while Double Dutch is already a team performance splitted into singles (DD3), pairs (DD4) or triad.
Each athlete or team then performs a choreography of 60 to 75 seconds showcasing their favorite skills and uniqueness. Judges watch the routine to annotate the difficulty, variation, musicality or deductions based on the scoring system. All the annotations made by judges contribute to the total score of the freestyle, ranking different competitors.
% TODO : footnote scoring system locally (belgium) vs international

\subsection{Wat are the challenges for judges today?}

% Zoals eerder vermeld verbeteren skippers op fysiek en technisch vlak. Skills variëren meer en worden moeilijker. Voor het onderdeel single rope moet iedere springer dezelfde truc uitvoeren, echter wordt het complexer bij double dutch, daar worden scores toegekend per 'snapshot'. Het jurylid moet draaiers, springers en touwsnelheid tegelijk in het oog houden. Elk van de drie vermelde aspecten levert potentiële levels op en kan elkaar snel opvolgen.

As described earlier, jumpers are improving quickly throughout the years. To elaborate this idea, it is mentioned in one of the IJRU 2023 livestreams that quadruple unders where 10 to 15 years ago rather unique, last years quintuple unders are becoming the norm. This is also true for Double Dutch freestyles or speed scores. With the current judging rule-set for double dutch, routines are replayed at half speed to increase judging accuracy on competitions.

\subsection{Which techniques can we apply to increase the accuracy of judging?}

The current panel consists of multiple judges (10-12), evaluating 5 different aspects of a routine at the same time. Mostly two or three people are judging the difficulty of the routine. As it was hard to keep up, the half-speed-replay was added to correctly assign a score.
Using the replay system, more time is needed to evaluate a routine, requiring additional difficulty certified judges on competitions. To put this further in perspective, a competition has 2 judging panels to improve competition flow and wait times. When the replay is applied, panel A could judge single rope, panel B double dutch where there are two diff panels within the second panel to alternate between watching live, replaying and counting the total score, while the other difficulty panel could be replaying, counting and the watching live. This results in around 1 judge for each 10 participating jumpers. Even more for team competitions.

Other than the replay system, different judging-rules can be applied, the current rule system in Belgium dates from 2023, requiring all judges to do a new exam to be certified again.

So splitting responsibility, fine-tuning the rule-set or replaying are already in use. The question arose with the rise of Artificial Intelligence, if machine learning could be applied to judge freestyles. We know AIs are not perfect, so are judges. Both can make mistakes while assigning levels or deductions to a routine.


% Om alles zo goed mogelijk te jureren worden double dutch oefeningen op A-competities herbekeken. Dit is echter limiterend, aangezien er meer juryleden nodig zijn of meer tijd per routine, maar dagen zijn beperkt.
% Om dit probleem op te lossen werd aan veel gedacht. Choreografieën op voorhand doorsturen is te veel voorbereiding voor het jurylid. Daarbij duurt het begrijpen van de beschreven skill vaak langer dan het visueel te zien.
% Daarnaast zijn eventuele wijzigingen in het jurysysteem weer belastend voor de springer en het jurylid. Immers wordt het geven van levels soms een reflex, dit kan voor foutieve resultaten zorgen bij wijzigingen.
% In het nieuws komen soms nieuwe A.I. tools die inspireren. Voorbeelden van beeldherkenning duiken op; zelfrijdende auto's, gebarentaal herkenning, emotiedetectie en de nextJump Speedcounter. Hierdoor de vraag; ``Zijn huidige computer vision technieken voldoende voor het herkennen van rope skipping skills?''

\subsection{How can we integrate machine learning into jump rope}

% TODO : footnote nextjump
The preferred solution in this proposal is an AI-model recognizing (sub)skills in a SR or DD3 freestyle (final decision has yet to be made). As of august 2023, NextJump tested their AI-speed-counter on the world competition acquiring, really accurate results.
Using this as a base and examples of human action recognition (har in sports, https://link.springer.com/article/10.1007/s10462-024-10934-9), the likelihood to succeed implementing skill-recognition in freestyles grows. % TODO : source

When skills are recognized, they can be mapped to their corresponding level and summed up to achieve the total score of a freestyle.

\subsection{Which data is available for the AI to use?}

Both individual and team freestyles are mostly recorded on competitions. All A-level team routines from the last three years (2022, 2023 \& 2024) are already available and provided by Gymfed (Belgium) for training. Additionally some clubs post their competition videos on social media which provides more data from the last 8 years and other camera perspectives.

To give more perspective, safely guessing DD3 data = 286-357 video's of length 1 to 1min15 (2022-2024 Belgium teams, virtual world contest \& two other world livestreams)
However, individual freestyles are more abundant. Calculating only my own recordings from 2024 will amount to a minimum of 782 freestyles (includes 2 camera perspectives, so about half of it are different routines). Additionally, collecting more individual freestyles will quickly amounts to a minimum of 1000 routines or a 16h+ of freestyle data, whereas DD3 freestyles are close to 5h. The NextJump speed counter used 10h of speed data to achieve their current results.

% TODO : update speedcounter 36h

%\begin{itemize}
%    \item \textbf{Machine learning heeft data nodig. Wat is beschikbaar en hoe kunnen we de accuraatheid van skills meten?}
%    De voorkeur voor een oplossing ligt bij het toepassen van machine learning op opnames van recente competities. Daarnaast zijn er tal van beelden te verzamelen van social media. Daarnaast kunnen eigen opnames van voorgaande en toekomstige wedstrijden de accuraatheid van het proefconcept verhogen.
    
%    \item \textbf{Wat moet het proefconcept kunnen?}
%    Het experiment moet de verschillende skills in een gegeven video benoemen en eventueel aangeven hoe zelfzeker het is. Hiervoor hebben alle trucs een eenduidige classificatie nodig die alle unieke combinaties van elkaar kunnen onderscheiden.
%\end{itemize}

\subsection{Which discipline will be focused on?}

Only one discipline for this thesis will be chosen.
There are two main disciplines to choice from. Either single freestyles (Single Rope, SR) or Double Dutch Single Freestyles (DD3)
Given the previous calculation and availability of data, it seems individual freestyles seem easier to get and the most likely path, however, in first thought, the variety and possible skills also increases, resulting in a more complex labeling (see skillmatrix later). One possibility for both DD3 \& SR is to omit certain special cases simplifying the entire problem in a minimum viable product (MVP). My educated guess based on experience would be more omitting for SR compared to DD3.

% TODO : Skill label example DD3 & Single freestyle at skillmatrix

\begin{table*}[t]
    \centering
    \begin{tabular}{|l|l|l|}
        \hline
        Area & SR & DD3 \\ \hline
        \#Freestyles & 782-1000+ & 286-352 \\ \hline
        Hours & 13h-16h+ & 5-6h \\ \hline
        Years & 782 in 2024 & mainly 2020-2024 \\ \hline
        MVP & basic variation elements & basic powers, gyms, turnerskills \\ \hline
        Level-guessing & 0 to 8 & 0 to 6-8 \\ \hline
        Theoretical level limit & 8+ levels possible & 8+ levels possible \\ \hline
        Variation elements & 6 & 4 \\ \hline
        skillmatrix & more complex & simpeler compared to SR \\ \hline
        longer sequences & / & / \\ \hline
        individuals & 1 & 3 \\ \hline
        competitions & Oct-Nov & March-Apr \\ \hline
    \end{tabular}
    \caption{Data comparison}
\end{table*}

Both datasets are skewed, there are common skills which can be found in 30-70\% of the routines (educated broad guess based on experience), where others occur only once, twice each competition, if they'll occur at all.

\subsection{What would be a minimal Proof of Concept (PoC)?}

The MVP would be a model recognizing the most common skills.
This means omitting or just marking special combinations, longer double dutch switches or longer time sequences of emptiness in general.
By starting with a simplified notation, hereafter called pointified notation, dotted notation or skillmatrix/table representation, all (sub)skills can be mapped to this general skillrecognition notation/table.
Afterwards, levels can be given to each (sub)skill in the matrix.
An educated, quick guess based on my own experiences would be that, at least, 85-95\% or more skills on competitions could be labeled using the simplified notation, not including special cases.

\subsection{Unanswered questions}

% Naast de inleidende deelvragen zijn er nog onbeantwoorde vragen. Hiervan vind je hieronder een oplijsting en worden verwerkt als antwoord in de resultaten of conclusie. Om daartoe te bekomen vind je in wat volgt een literatuurstudie naar de recente technieken die we kunnen gebruiken voor het herkennen van skills en de uitdagingen ervan. Nadien volgt het plan van aanpak, de verwachte resultaten en tot slot de conclusie, maar eerst de deelvragen.

% \begin{itemize}
%    \item Wat is het beste model om de trucs te herkennen?
%    \item Hoe wordt het model opgebouwd?
%    \item Wanneer zijn voorspellingen goed genoeg om te dienen als extra jurylid of als steunmiddel?
%    \item Kunnen we de AI-Judge gebruiken om juryleden te verbeteren?
%    \item Indien het proefconcept een oefening kan jureren, kunnen we hieruit afleiden dat dit voor alle jurysporten kan? Indien wel, wat zou de beste aanpak zijn?
%\end{itemize}

Until now, only the scope has been narrowed. The literature, implementation \& labeling will give more clarification on different questions.

\begin{itemize}
    \item How will the model be built?
    \item What would be the main structure of the model?
    \item Which human activity recognition examples can be used or altered as the base of the model?
    \item When are AI-recognitions acceptable to potentially use on competitions?
    \item How much data is expected to increase the accuracy off the Judge.
    \item How can we use the AI-Judge to improve judges?
    \item What needs to changed to a working model, to apply it on other judgesports such as gymnastics, synchronized swimming...
\end{itemize}

%---------- Stand van zaken ---------------------------------------------------

\section{Literatuurstudie}%
\label{sec:literatuurstudie}

% TODO: (TAAL controle literatuur)

% TODO BP: volgorde en uitbreiding literatuur.
% Convolutielagen in een nutshell -> van matrix naar matrix (afbeelding), maar cellen van een nieuwe matrix zijn een som van andere rijen en kolommen (initieel pixels)
% LSTM in een nutshell -> geheugencellen, activatiefunctie -> input, memory, activativering output.
    
Het onderzoek naar skillherkenning wordt stapsgewijs opgebouwd. Beginnen doen we met een introductie en definitie van skills, waarna gepast onderzoek naar technieken kan gebeuren. Stilaan bouwen we dit op tot we beter definiëren waar de moeilijkheden liggen bij het herkennen van skills om zo de gevonden modellen te verfijnen tot het gewenste resultaat.

\subsection{Skills intro}
\label{subsec:basisskills}

% Eerder beschreven we al de aanwezigheid van verschillende onderdelen in Rope Skipping. Om het onderzoek doenbaar te houden, leggen we de focus op single rope individuele freestyle (SR) en double dutch single freestyle (DD3). Beide onderdelen hebben een ander jurysysteem, waarbij verschillende elementen terugkeren, maar ook anders zijn. Soms wordt ook enkel over SR gesproken om niet in herhaling te vallen.

Earlier we described the presence of multiple disciplines in Jump Rope. To keep the research doable, freestyle skill recognition will be start on one discipline. The main candidates for freestyle skill labeling will be single rope and double dutch single freestyles. Bot judging systems incorporate their own rules, where overlap/similarity of skill movements are common. 

\subsubsection{Single Rope - SR}

Single Rope is the style defined where jumpers use their individual rope to execute different skills belonging divided in categories: gymnastics G, powers P, wraps W, releases R, backwards B, multiples M, crosses C, footwork F, Interactions I and an endless possibility of combinations. Multiples for example are actually multiple crosses in one jump. Likewise, releases can be combined with wraps, multiples with powers and so on.

Some examples:

% TODO : bijlage bij BP
\begin{itemize}
    \item Crosses
    \begin{itemize}
        \item side swing (s - swing next to the body)
        \item open (o - normal jump)
        \item cross (c - crossed arms on the stomach)
        \item toad (raise knee, other arm under your knee)
        \item EB (arm on stomach + arm on back)
        \item AS (arms crossed behind the knees)
        \item CL (one arm behind the knees, other on the back)
        \item TS (arms crossed behind the back)
    \end{itemize}
    \item Multiples
    \begin{itemize}
        \item double - DU - 2 rotations
        \item triple - TU - 3 rotations
        \item quad - QU - 4 rotations
        \item quint - 5 rotations
        \item TU.s.EB.o - first rotation on the side, next EB, next open
    \end{itemize}
    \item Powers
    \begin{itemize}
        \item push-up - to plank position and pushing upwards while pulling the rope underneath your feet.
        \item split - idem push-up, but to split
        \item frog - handstand
    \end{itemize}
\end{itemize}

Each skill will then be assigned a level contributing to a score. 

\subsubsection{Double Dutch}
\label{subsubsec:double dutch}

Double Dutch consists of two turners and one jumper alternating rope rotations. Elements in Double Dutch are similar, but different at the same time. The jumper does all the skills, mainly powers, gymnastics or footwork, since he doesn't need to hold the rope. Turners can manipulate the rope using multiple unders, turnerskills like crossed arms, EB, toad\dots or even involve gymnastics themselves.
To judge double dutch, snapshots are taken, then the level will be given depending on the combination of turnerskills, jumperskills and roperotations.
Switching turners also contributes to scoring.

Like any discipline, mistakes can happen, they'll be deducted from the total score.

\subsection{Computer vision}
\label{subsec:computer vision}

% TODO : footnote refer to table above for numeric data
To recognize the skills automatically input data is needed. As there are quite some videos already on socials, as well as in-house recordings, the choice of learning from videos was made quickly. The field of study in which computers recognize features, people or other objects in digital imagery is called computer vision. More specifically, the focus is recognizing human actions in these recordings, called human activitity recognition or HAR \autocite{Pareek_2020}.

Human Gait Recognition, HGR, or Human Pose Estimation, HPE, are also used to recognize human activities. Although the three techniques are closely related, each has a different nuance. Gait recognition looks at a person's typical movements, gestures or behavioral patterns \autocite{Alharthi_2019}, while pose estimation looks specifically at poses or special expressions \autocite{Song_2021}. They also talk about how pose recognition, skeleton-based, can be used as a tool are to improve activity labeling.

Models providing skeleton-structure or others can be used to improve action recognition, mainly because crosses, powers, releases\dots are build from these poses.

\subsection{HAR general progress}

% TODO : find paper, reference to come to steps or use pareek again?
The following approach is mainly based on the survey of \textcite{Pareek_2020}, he describes the recent updates in human activity recognition. 

\begin{enumerate}
    \item Jumper localization
    \item Action segmentation, start of skill, end of skill
    \item Predict level or variation element (multiple, cross, gym...)
    \item Predict the effective skill
\end{enumerate}

Jumpers can stand everywhere in the field, in order to perform better, locating and cutting out the athlete can improve the segmentation model. When the skippers are centered, better action segmentation can be performed. This allows predicting skills on newly recorded videos, without needing to cut out the different skills.

\subsection{Jumper localization}
\label{subsec:jumper localization}

% Het eerste doel van het proefconcept is het herkennen van de springer. \autocite{Zaidi_2021} deed een onderzoek naar de moderne deep learning technieken voor het herkennen van objecten en besprak verscheidene modellen, zoals YOLO, SSD, CenterNet, CNN, hun varianten en de nood aan compactere modellen voor dagelijks gebuik. Echter vergt het onderzoek niet het herkennnen van rope skippers op één afbeelding, maar gedurende de gehele video of videofragmenten. Om dit proces mogelijk te vereenvoudigen vergelijkt \textcite{Gao_2022} recente deep learning technieken die de voor- en achtergrond in video's van elkaar onderscheiden, genaamd video object segmentation of VOS. Gaande van het herkennen van een persoon, tot het herkennen van armen, benen, voeten, handvaten en touwen. Hoewel \textcite{Bharadiya_2023} aangeeft dat CNN kenmerken herkent ongeacht de positie in de afbeelding, zouden we de gemaskeerde beelden kunnen schalen en centreren. Vooral bij wedstrijddata springen atleten dichtbij of ver in het veld ten opzichte van de camera. In het volgende deel kunnen we dit dan gebruiken voor volledige trucherkenning.
    
% TODO : general source? https://arxiv.org/pdf/1905.05055
Many research towards image recognition has been done. The best models mostly utilize Convolutional Neural Networks pertaining spacial information in the image \autocite{Zaidi_2021}. In their paper, they compare some recent models for object detection such as YOLO(v4), CenterNet, SSD, EfficientDet-D2, each using some backbone architecture like VGG-16, AlexNet, GoogleNet or lightweight models such as ShuffleNet or MobileNet (all using CNN's). Some of them being real-time models (fps > 30).
The goal of localizing the jumper is to center the athletes in the middle of the screen/video. \textcite{Bharadiya_2023} elaborates that the position of objects in images doesn't really matter, but their are no clear statements about the size of objects. It could be that a jumper takes up 80 percent of the screen, while moments later he moved backwards and only takes up 30 percent of the video. Instincts tell us that centered and scaled data will work better later.

One of these models can be taken as a base, using transfer learning to fine-tune the results to localize the jumper. % TODO : footnote bharadiy explains transfer-learning

To improve localization, video object segmentation or video instance segmentation can be used. \textcite{Gao_2022} lists some object segmentation models, like SwitNet using ResNet18 as a backbone to be the fastest.
Other possibilities would be Cutie % TODO : source
or DensePose % TODO : source + image
or meta sam2 % TODO : source

Densepose seems to be able to give the bounding boxes of the main poses detected. Perhaps just a convex hull and some padding will be enough for smart cropping and you will not need to train a network from scratch. However, a local tryout (without boxes) was rather slow (1.2 fps on GPU within a Docker container).

As a video is basically sequence of images, some smoothing can be applied to the guessed boxed, improve the next parts.

% TODO : VGG-16? Rather expensive calculations

\subsection{Video action segmentation}

When we have our jumpers centered, we can try splitting the video in (sub)skills, because splitting new videos manually takes to much time and is impractical on competitions.
A model like LTContext % TODO source
or others would be appropriate. % TODO : footnote paperswithcode
Just like the localization, denseposing, extracting poses, foreground, background\dots would improve the segmentation.

Another interesting approach was \textcite{Zaidi_2021} modifying the LSTM-model to incorporate longer time sequences, to predict a gymnastic score of a full routine.

\subsection{Skillherkenning}
\label{subsec:skillherkenning}

translate 2.8

To recognize actions in videos, lots of research has been done. % TODO : that general source
% TODO sportHAR ref
sportHAR clearly states the evolution from normal CNN architectures, to recurrent neural networks for time sequences, remembering context, like the Long Short Term Memory model (LSTM). Then combining CNN output into LSTM or even implementing a convolutional filter in the memory cell \autocite{Shi et all}. % TODO : source

\textcite{Wang_2019} investigated an improvement for current convolutional or recurrent models. They found that LSTM memory cells were too simple to contain higher-order complexities. As a result, they designed a memory in memory component, to replace the previous cell, which could predict actions on complexer data sets. This quickly formed the name Memory in Memory, MIM. Later, \textcite{Lin_2020} used
a self-attention memory cell inside the convLSTM that can memorize global aspects in time and space. With about 35\% the number of parameters compared to Wang's MIM-model, SAM achieved a similar score as on the moving MNIST dataset, but faster. Although the focus of these papers was predicting future actions, the output can again be transformed to a classification rather than a prediction.

Another approach would be using transformers as a described possibility in % TODO : sportshar.
ViT-TAD yang m. (https://arxiv.org/pdf/2312.01897), Swin transformer or VideoMAE v2 seem good options, but further research/try-outs will be required to ensure the models can predict actions.

So using convLSTM, SAM or a transformer brings us to how exactly we want to predict.

% TODO : find other model besides sam that is not a transformer to try-out, as SAM is already convLSTM improved.


%\subsection{SAM \& MIM}
%\label{subsec:SAM&MIM}

% \textcite{Wang_2019} onderzochten een verbetering voor de huidige modellen voor tijdruimtelijke data. Ze vonden dat LSTM-geheugencellen te simpel waren om complexheden van hogere orde te bevatten. Hierdoor ontwierpen ze een memory in memory component, ter vervanging van de normale geheugencel, die datasets met een hogere moeilijkheidsgraad beter konden voorspellen. Hieruit vloeide de naam Memory in Memory, MIM.
% \textcite{Lin_2020} daarentegen gebruiken een self-attention memory cell binnenin het convLSTM die globale aspecten in de tijd en ruimte kan onthouden. Met ongeveer 35\% het aantal parameters t.o.v. Wang's MIM, bereikte SAM een gelijkaardige score als op o.a. de movingMNIST dataset, maar sneller. Hoewel de focus van deze papers lagen op het voorspellen van toekomstige acties, kan de output weer omgevormd worden, naar een classificatie i.p.v. een voorspelling.


\subsection{Skill complexity \& levels}
\label{subsec:skillcomplexiteit}

A basic explanation of skills was given earlier, with some examples. In order to recognize all skills, they should be worked out as well as possible. This section explains the basics of the level system based on Gymfed judging rules 2024-2025.
A cross, toad, or crouger are “forward” skills performed without a preparatory rotation, there are no restrictions behind the body requiring a preparatory rotation. Tricks for which this does apply are AS, CL, TS, or meghan, for example, they consist of going to and coming out, to perform the full skill. So we label both parts as a sub-skill. Scores are given based on levels, so you get a level for each restriction of the arms. For example, a toad, crouger and EB all get you 1 level, while a TS, elephant or CL gives you 2 levels.
Ordinary crosses are too simple and yield nothing or only half a level and are otherwise not additive to combinations. If you frequently combine such basic exercises into a sequence, you get the level for each unique transition. For example;


\begin{itemize}
    \item toad-crouger-toad (1-1-1)
    \item toad-toad-toad (1-1-herhaling)
    \item CL (2) and later toad-CL (1-2) although you need 3 rotations (toad, go-to-CL, come-from-CL)
\end{itemize}

Arm switches, called switch crosses, gets you bonus points (1 or 2), marked with an x.

\begin{itemize}
    \item toad-xtoad (1-2)
    \item AS-xAS (2-4)
    \item AS-CL (2-2)
    \item AS-xCL (2-3)
\end{itemize}

All these crosses can be combined with other categories, even with pair interactions, you can apply crosses, take someone in a toad, cross or TS.
Multiples are built from unique rotation combinations. Some quads as example:

\begin{itemize}
    \item QU.s.o.AS.o (QU = 3, AS = 2, total = 5)
    \item QU.s.AS.o.o (5)
    \item QU.s.c.AS.o (5)
    \item QU.s.toad.AS.o (6)
    \item QU.s.EB.CL.o (5 or 6 depends if your EB-arm sticked to your back)
\end{itemize}
% TODO: \usepackage{graphicx} required

\begin{figure}
    \centering
    \includegraphics[width=0.7\linewidth]{img/doubledutch-matrix}
    \caption[skillmatrix-DD]{Representation of a skillmatrix used for Doube Dutch}
    \label{fig:doubledutch-skillmatrix}
\end{figure}

In addition, these skills can be combined with powers, releases\dots, and there are a number of special cases or exceptions that earn fewer or extra points. (Omitted in the MVP)
Put it all together, to create a skill matrix (work in progress), see double-dutch-matrix fig\ref{fig:doubledutch-skillmatrix}.

The skillmatrix for SR is still a little in progress since last year, knowing that some special cases can be omitted.

%\subsection{Skillafbakening \& branching}
%\label{subsec:video segmentation & branching}

% In \textcite{Zahan_2023} wordt het standaard LSTM uitgebreid om beter langdurige informatie te verwerken om resultaten van gehele gymnastiekoefening te voorspellen. Aangezien het jurysysteem van die data hierdoor één-op-één gekoppeld is aan het model, is dit niet robuust genoeg voor veranderende versies. Het proefconcept wenst dus de skills te leren waarop de levels gemapt worden via een functie.
    
% Om alle trucs en combinaties van elkaar te onderscheiden werd stap per stap een skillmatrix opgebouwd. Deze matrix tracht zo veel mogelijk, om niet te zeggen alle, oefeningen op te delen volgens kenmerkende eigenschappen. Dit denkproces bracht de afbakening van skills en combinaties in kaart.
% Freestyles en ander beeldmateriaal moet gesplitst worden in eenduidig, atomair benoembare (sub-)skills, minivideo's, die slechts één representatieve voorstelling hebben in de skillmatrix.
% Concreet betekent dit dat na afbakening van skills, iedere mini-video, nog herstelbaar moet zijn tot zijn combinatie.
% Doorgaans geeft het verlaten van de grond aan dat een truc begint, analoog bij het landen en in de lucht zijn. Dit is echter niet altijd het geval bij wraps of releases. Tevens moeten de geknipte video's, na de afbakening van de trucs, even lang zijn (gelijke dimensies). Een simpele truc hiervoor is het toevoegen van \emph{lege} informatie, zwarte frames, genaamd padden (Engelse term). Vaak zullen langdurige intervallen wraps of herstel van fouten zijn. Deze lange sequenties kunnen apart gemodelleerd worden of verder opgesplitst om de padding niet te lang te maken.

\subsubsection{Skillmatrix}

De de skillmatrix is voor verandering vatbaar, wanneer combinaties ontdekt worden die er niet in passen en ziet er op dit moment voor single rope als volgt uit:

\begin{itemize}
    \item \textbf{Description}: solely informative
    \item \textbf{Pointified notation}: Own notation for unique naming of skills
    \item \textbf{Category}: I, G, P, W, R, B, M, C, F
    \item \textbf{People}: skippers executing the skill (could be 2 or 3 in a team exercise)
    \item \textbf{mistakes}: skippers that made a mistake
    \item \textbf{Rotations}: rope rotations, 0-8
    \item \textbf{Rope direction first rotation}: forward, backward, side
    \item \textbf{Rope direction first rotation}: idem
    \item \textbf{Rope rotations switches}: 0 of 1, theoretical 2, 3, 4 possible
    \item \textbf{PBR}: 0,1,2+ parallel body-rotations with the ground (e.g. crab, push-up)
    \item \textbf{TTs}: 0.25 -> 2+ 'turntables' (e.g. hummingbird, 360, halve draai, turn table crab, turn table pushup)
    \item \textbf{multiplepart} 8x duplicated, one for each rope rotation, wrapstatus\dots
    \begin{itemize}
        \item \textbf{element} s, o, c, toad, EB, AS, TS, knee, straddle, mic, 1-H-release, powers, gyms \dots
        \item \textbf{switch} xAS, switch cross
        \item \textbf{wrapped bodyparts}: arm, leg, between leg around arm...
        \item \textbf{Interaction type}: scoop, take forward, take backward
    \end{itemize}
\end{itemize}

    
Skillmatrixes can change over time, initial skills not fitting the matrix will be left out for the MVP.
Furthermore, you see some labels can happen one at a time (cross-type), others (e.g. category) can happen at the same time (multi-branch output as described in section 4.4.1 \autocite{Coulibaly_2022}), others require softmax.

\subsection{Group activity}

As DD3 is a group activity, problems could arise while trying to detect skills. However, the hypothesis is that a DD3 freestyle always acts as one unit, thus not really requiring much special attention. This could pose a problem when adapting SR to SR2 where two individuals are not exactly one unit. Some further research can be done in models like stagNet (\autocite{stagnet-volleybal}) fix this.

\subsection{Challenges}
\label{subsec:challenges}

Unknown skills or special cases pose a problem. That's why the skillmatrix needs to defined in such a way that new combinations can fit the matrix as much as possible and/or in combination with zero-shot learning. (Sort of marking unique skills as 'I don't know' so that others new/unique ones will also be marked as 'I don't know')


\subsection{Summary literature}
\label{subsec:summary literature}

The PoC needs to localize the jumper, using existing models fine-tuned, segment actions, preferably through transfer-learning and using own splits or guessed splits to predict (sub)skills.
Predicted skills will then be mapped to their corresponding level.

% Refereren naar de literatuur kan met:
% \autocite{BIBTEXKEY} => (Auteur, jaartal): voor een referentie tussen
% haakjes, waar de auteursnaam GEEN onderdeel is van een zin.
% \textcite{BIBTEXKEY} => Auteur (jaartal): voor een narratieve referentie,
% waar de naam van de auteur effectief een onderdeel is van de zin.


%---------- Methodologie ------------------------------------------------------
\section{Methodologie}%
\label{sec:methodologie}

Aan dit proefconcept kan rustig gewerkt worden in de zomer en veel in het tweede semester, (geen stage, enkel BP, ITP\&CO, Data science en big data). Aangezien het labelen veel tijd in beslag kan nemen, zal dit iteratief en semi-supervised gebeuren doorheen het onderzoek, gefocust op individuele skills (SR) en double dutch single freestyles (DD3). Hierdoor zal slechts een deel van de reeds verzamelde 500GB aan beeldmateriaal gebruikt worden. Echter is dit niet het enige beschikbare materiaal. Een ruwe schatting denkt dit aantal nog makkelijk te verdubbelen in de toekomst door online te zoeken en rond te vragen binnen de rope skipping community.
    
Verder wordt het project iteratief opgebouwd. Een eerste diepere duik in de wereld van deep learning werd reeds gemaakt, zoals gezien in de literatuurstudie. Deze zal regelmatig opknapbeurten of aanvullingen krijgen door opduikende moeilijkheden en/of problemen.

\subsection{Iteratief proces}
Om de proef te starten wordt de nieuwe kennis toegepast op een beperkte set video's om deze te bakenen in mini-video's. Nadien wordt de borstel door de huidige data gehaald om er later beter op te kunnen filteren. Zo kunnen we ook statistieken maken die een kleine vergelijking geeft tussen wedstrijddata vs andere data, meest gesprongen truc, club, land, jaar met meeste data\dots
Als volgende moeten de modellen meer performantie bieden, dit kunnen we doen door de dimensionaliteit te verkleinen via en de best mogelijke techniek(en) voor videoherkenning en beter rekening houden met meerdere springers al dan niet door gebruik van één of meerdere VOS technieken. Hierdoor kunnen we de skillafbakening verbeteren en omtoveren tot een semi-supervised model dat stilaan zelf extra trainingslabels voorziet, met verificatie en aanvullingen van trucs.
    
Losse stukken code en probeersels maken de code snel onoverzichtelijk, foutgevoelig en niet bevorderend om te analyseren. Daarom wordt een speciaal moment voorzien die de cohesie, portabiliteit en structuur van de code verbetert. Hieruit volgt dat nadien de volledige focus gelegd kan worden op implementatie van een eerste HAR-model, verfijning en vergelijking met andere modellen om het beste eruit te halen. Tot slot wordt het onderzoek aangevuld met alle de bevindingen, resultaten en de conclusie.
    
In figuur~\ref{fig:../graphics/gantt} wordt een schatting van de tijdsplanning gegeven en in de oplijsting hieronder vind je meer details per fase.

\begin{figure}
    \centering
    \includegraphics[width=.8\columnwidth]{img/gantt}
    \caption{\label{fig:gantt}Indicatie tijd en indeling per fase}
\end{figure}

\subsection{Fasering}

\begin{itemize}
    \item \textbf{Fase 0: Verbeteren onderzoeksvoorstel}
    \begin{itemize}
        \item \textbf{Doelstelling}: Kwaliteit verhogen
        \begin{itemize}
            \item Structuur begint aan te voelen als: een novel aanpak voor HAR in jurysporten
            \item of
            \item Toepassing van recente HAR technieken op complexere rope skipping data
        \end{itemize}
        \item \textbf{Aanpak}:
        \begin{itemize}
            \item Meer vorm geven aan try-outs en effectieve uitvoering
            \item Extra literatuurstudie, op basis van extra ondervindingen
            \item Blijven bouwen aan dat proefconcept
        \end{itemize}
        \item \textbf{Must have}: Verhoogde kwaliteit (literatuur, proeven)
        \item \textbf{Should have}:
        \begin{itemize}
            \item semi-supervised border labeling
            \item meer kennis over self-attention, lstm memory cells
        \end{itemize}
        \item \textbf{Could have}:
        \begin{itemize}
            \item Een eerste HAR model
            \item afgewerkte HAR- \& dimensionaliteit-literatuur
        \end{itemize}
        \item \textbf{Will not have}: Nieuwe eigen modellen
    \end{itemize}
    \item \textbf{Fase 1: Basic probeersel skillgrenzen}
    \begin{itemize}
        \item \textbf{Doelstelling}: Mini model, standaard CNN en/of LSTM, die door een tiental video's kan proberen om frames aan te duiden die de grens van een skill zijn.
        \item \textbf{Aanpak}:
        \begin{itemize}
            \item Eigen testvideo's verder labelen
            \item Data preprocessing - opsplitsen video's
            \item Train \& test
            \item Bouwen/zoeken visual die de video-opdeling van een nieuwe testvideo kan tonen
        \end{itemize}
        \item \textbf{Resultaat, deliverable(s)}:
        \begin{itemize}
            \item \textbf{Must have}: skillsafbakening in video als oplijsting van framenummers
            \item \textbf{Should have}: Verhoogd begrip computer vision \& skillafbakening
            \item \textbf{Could have}: modelkeuze (LSTM, CNN, YOLO...)
            \item \textbf{Will not have}: Opgeslagen mini-video's
        \end{itemize}
    \end{itemize}
    \item \textbf{Fase 2: Cleaning \& ordenen huidige RS data - database}
    \begin{itemize}
        \item \textbf{Doelstelling}: Opkuisen structuur en huidige video's
        \item \textbf{Aanpak}:
        \begin{itemize}
            \item Aanmaken databank waarbij elke video 'extra' info krijgt, wie, wat, waar, kwaliteit, obstructielevel/ruis, competitie/voorbeeldoefening, landscape/portret, bestandslocatie... (om dubbels te vermijden)
            \item Data van Arne (groot bk, bk 23 en bk 24) herbenoemen
            \item Eigen opnames verder knippen en benoemen
            \item Gescrapete video's herbenoemen
            \item DB voorbereiding skilltabel (mogelijke skills, makkelijk te vinden in video)
            \item DB voorbereiding skillborders voor bevestigde of semi-supervised splits + skill\_key
        \end{itemize}
        \item \textbf{Must have}: alles
    \end{itemize}
    \item \textbf{Fase 3: Dimensionaliteit}
    \begin{itemize}
        \item \textbf{Doelstelling}: Aanvullen literatuurstudie m.b.t. dimensionalieitreductie voor computer vision video machine learning
        \item \textbf{Aanpak}:
        \begin{itemize}
            \item Zoektocht voor specifiek HAR
            \item Lijst van technieken of het best mogelijke
            \item Voorbeeld met convLSTM, SAM of MIM?
        \end{itemize}
        \item \textbf{Must Have}: Minstens één techniek
        \item \textbf{Could have}: Voorbeeld met HAR
        \item \textbf{Will not have}: gecompresseerde video's
    \end{itemize}
    \item \textbf{Fase 4: Literatuur video object segmentation \& multi actors}
    \begin{itemize}
        \item \textbf{Doelstelling}:
        \begin{itemize}
            \item Literatuur multi actors verbeteren
            \item Literatuur VOS longlist technieken
            \item Invloed multi actors op VOS weten
            \item Shortlist van VOS technieken die omkunnen met meerdere actoren
        \end{itemize}
        \item \textbf{Aanpak}:
        \begin{itemize}
            \item Huidige source, multi actoren daarvoor beter doornemen
            \item VOS Technieken zoeken die hiermee om kunnen
            \item Armen, benen, touw, handvaten of enkel het geheel?
            \item Shortlist van maken
        \end{itemize}
        \item \textbf{Must have}: Shortlist VOS
        \item \textbf{Should have}: Techniek die het touw accuraat in beeld brengt
        \item \textbf{Will not have}: 'aparte' video's die door per actor door het HAR-model moet
    \end{itemize}
    \item \textbf{Fase 5: Bijlagen en methodologie}
    \begin{itemize}
        \item \textbf{Doelstelling}: Bijhouden methodologie en bijlagen opstellen
        \item \textbf{Aanpak}:
        \begin{itemize}
            \item Volgen van deze methodologie
            \item Afwijkingen noteren
            \item Bijlagen toevoegen, skillist, skillmatrix met foto's
        \end{itemize}
    \end{itemize}
    \item \textbf{Fase 6: Skillborders deel 2 (semi-supervised)}
    \begin{itemize}
        \item \textbf{Doelstelling}: Verbeteren en vereenvoudigen eigen werk
        \item \textbf{Aanpak}:
        \begin{itemize}
            \item Overschakelen op semi-supervised
            \item Verbeteren probeersel skillgrenzen, m.b.v. VOS \& dimensionaliteitsreductie
            \item Toevoegen grenzen testvideo's - manual, verified, semi-supervised
            \item Zo zorgen testen tussendoor voor extra labels
            \item Toevoegen andere modellen YOLO, SSD...
        \end{itemize}
        \item \textbf{Must have}:
        \begin{itemize}
            \item split video optie
            \item verifieer grenzen (alle, volgens bepaald zekerheidspercentage)
        \end{itemize}
        \item \textbf{Should have}: /
        \item \textbf{Could have}: Verifieer niet bevestigde labels en toon grootste onzekerheden
        \item \textbf{Will not have}: Niet alles moet gelabeld zijn
    \end{itemize}
    \item \textbf{Fase 7: Procesautomatisatie}
    \begin{itemize}
        \item \textbf{Doelstelling}: Verhogen weergave, keuzeopties modellen, begrip en staat van het proefconcept
        \item \textbf{Aanpak}:
        \begin{itemize}
            \item Configuratiebestanden: run info, modelselection, keuze labels train...
            \item Portabiliteit: Docker?
            \item Logs, run en modelstatistieken, datastatistieken
            \item Documentatie
            \item Toevoegen nieuwe video, geeft skillgrenzen en verifieeropties
        \end{itemize}
        \item \textbf{Must have}: portabiliteit, configuratie
        \item \textbf{Should have}: toon meest onzekere niet-geverifieerde skills
        \item \textbf{Could have}: statistieken
    \end{itemize}
    \item \textbf{Fase 8: Literatuur HAR deel 2 (september/oktober)}
    \begin{itemize}
        \item \textbf{Doelstelling}: Extra zoektocht naar HAR-modellen, op het einde van de zoektocht kwam (Lin, 2020) tevoorschijn, beste paper voor HAR tot nu gevonden, maar dat wil zeggen dat er nog 4 jaar aan evolutie kan zijn
        \item \textbf{Aanpak}: zoeken
        \item \textbf{Should have}:
        \begin{itemize}
            \item Extra modellen
            \item Aangepaste shortlist
            \item recenter model
            \item generiek idee hoeveel literatuurstudie er nog bijkomt
        \end{itemize}
    \end{itemize}
    \item \textbf{Fase 9: Eerste HAR model (target september)}
    \begin{itemize}
        \item \textbf{Doelstelling}: Eerste implementatie HAR-model
        \item \textbf{Aanpak}:
        \begin{itemize}
            \item Implementatie van de code van het gekozen model
            \item Volledig begrip van het gekozen model. (Niet blackboxen)
            \item Preprocessing video's, splitsen, VOS, padding
            \item branching vs binaire classificatie voor alle kolommen
        \end{itemize}
        \item \textbf{Must have}: trucvoorspelling
        \item \textbf{Should have}: Keuze optie branch of binair
        \item \textbf{Could have}: Procentuele modelstatistieken (per testrun)
    \end{itemize}
    \item \textbf{Fase 10: Uitbreiden HAR}
    \begin{itemize}
        \item \textbf{Doelstelling}: Verbeteren model \& vergelijkingen
        \item \textbf{Aanpak}:
        \begin{itemize}
            \item Andere v/d twee als optie in config toevoegen: branching of niet
            \item Toevoeging gewenste config opties discovered along the way
            \item Toevoegen modellen SAM, MIM, convST-LSTM
            \item Semi-supervised skills aan trainingsset DB (verified or not)
        \end{itemize}
    \end{itemize}
    \item \textbf{Fase 11: bachelorproef scriptie}
    \begin{itemize}
        \item \textbf{Doelstelling}: Afwerking bachelorproef
        \item \textbf{Aanpak}:
        \begin{itemize}
            \item Bijwerken methodologie
            \item Methodologie finetunen (tussendoor bijhouden)
            \item Conclusie
            \item Dankwoord
            \item Bijlagen finetunen
            \item Inleiding en abstract finetunen, die overeenkomen met literatuur en resultaat
        \end{itemize}
        \item \textbf{Must have}: Alles
    \end{itemize}
\end{itemize}


\subsection{Aanpak beantwoording deelvragen}
\label{subsec:methodologie-deelvragen}

% TODO : finetune this : taal

Om antwoord te geven op de vragen; ``Wanneer zijn voorspellingen goed genoeg om te dienen als extra jurylid of als steunmiddel?'' en ``Kunnen we de AI-Judge gebruiken om juryleden te verbeteren of omgekeerd?'', moet er even samengezeten worden met juryexperten.
Een eigen voorstel is; Na indiening van de score, krijgt het jurylid de levels van een werkend model, die controleert en duidt die levels aan die niet overeenstemmen met de eigen gegeven levels, juryleden kunnen zichzelf ofwel verbeteren, ofwel het model verbeteren. Later kan een overgang gemaakt worden naar meest onzeker skills van het model verifiëren.

De deelvraag over de veralgemening tot jurysporten zal moeten afwachten op het verloop van het proefconcept.

\subsection{Training \& Hardware}

Werken met beeldmateriaal alleen al vraagt veel resources, laat staan het trainen op de data met een normale laptop. Best wordt er gewerkt met één of meerdere GPU's om het onderzoeksproces te bevorderen. Tussendoor kunnen berekeningen gemaakt worden om in te schatten hoe lang trainsessies zullen duren. Tevens geeft dit ook toekomstige referentie voor andere computer vision concepten.


%---------- Verwachte resultaten ----------------------------------------------
\section{Verwachte resultaten}
\label{sec:verwachte-resultaten}

    Het PoC is veel werk, in het goede geval wordt alles perfect afgewerkt. Het is uitbreidbaar en in te krimpen. Zo kan SR of DD3 overgeslagen worden tot slechts één onderdeel of kan de skillafbakening meteen zorgen voor problemen. Echter bleek uit de literatuurstudie dat dit wel moet lukken. Ook voor het herkennen van trucs zelf worden positieve resultaten verwacht. De hybride modellen van X, Y, Z en anderen tonen vaak resultaten van +90\% op bekende datasets zoals WISDQM of .... Hoewel de ropeskipping data veel gevarieerder is en wat complexer worden toch scores van +85\% verwacht met het SAM model van X.
    
    Wat het beste model is om de trucs te herkennen en hoe het geheel is opgebouwd, is nog onbekend. De verwachting is dat SAM of het convST-LSTM de beste resultaten behaald. Of het totale idee semi-supervised, CNN-afbakeningen, SAM/convST-LSTM skills met behulp van Docker overeind blijft, zullen we over een jaar weten.
    
    Wanneer de voorspellingen goed genoeg om te dienen als extra jurylid of als steunmiddel wordt op meer dan 95\% geschat en wanneer die goed kan aangeven dat hij het niet weet.
    
    Kunnen we de AI-Judge gebruiken om juryleden te verbeteren?
    Ja, er ontstaat uiteindelijk een verzamelde en gelabelde database, met vele oefenvideo's om uit te leren. Daarbij zou er in de database gefilterd kunnen worden op skill om makkelijk en een eenduidig antwoord moeten geven op de vraag: ``Wat is het level van ...?''
    
    Indien een AI een oefening kan jureren, kunnen we hieruit afleiden dat dit voor alle jurysporten kan? Indien wel, wat zou de beste aanpak zijn?
    Volgen van het semi-supervised model, een goede beoordelingsmatrix toevoegen, iteratief data toevoegen \& labelen, configuratie kiezen.


\section{Verwachte conclusie}%
\label{sec:conclusie}

    Juryleden kunnen verminderd en/of het werk van het jurylid vereenvoudigd. Daarbij kan het ingezet om nieuwe juryleden op te leiden.
    
    Waar resources zijn kan het, computationeel kan het nog beter en stilaan zouden de resultaten verhoogd kunnen worden. Dit kan m.b.v. betere modellen of meer data.
    
    Kan eventueel toegepast worden op andere jurysporten met gelijkaardige routines zoals gymnastiek, kunstschaatsen, acro of dressuur.
    
    Volgende stappen zijn het trainen en beoordelen van DD4, SR2 of SR4 waarbij iedere springer gelijktijdig de skill moet uitvoeren om punten te krijgen. Dit is dan weer meer gelijkaardig aan synchroonzwemmen, diving, of acrobatic tumbling.
    
